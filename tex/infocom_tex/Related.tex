\section{Related Work}
\label{sec:related}

The security of location data has been closely investigated in the previous literature~\cite{fawaz2014location,gibler2012androidleaks}, as such data can be used to identify a wide range of activities and valuable information of the carrier for tracking purposes~\cite{farrahi2010probabilistic}. For example, previous research has demonstrated that it is possible to reconstruct transportation modes of users through their location data alone~\cite{bierlaire2013probabilistic}. Meanwhile, recent rises on attacks on servers and information leaks indicate that we could not trust the service providers for keeping vital data secure~\cite{vasek2016hacking}. Therefore, we should find ways to store and process location data in a secure and trustworthy manner.

Previous work on keeping location secure has investigated multiple directions. The first direction, which is called k-anonymous obfuscation~\cite{casino2015k,phan2015kur}, tries to hide the true locations of users by obfuscating them to the granularity of larger cells. Such methods, although protecting privacy, make it harder to develop applications that require the precise locations of users. Another method is through statistical methods~\cite{seidl2015spatial}, which add random noise to the samples of individual users, but keep the global statistical parameters to be more or less reliable. Again, such methods are only suitable for large-scale statistical needs but are not useful where one user's data needs to be exploited for application needs, e.g., ride sharing, targeted promotions, and navigations. 

There has also been work to study the private equality test~\cite{kotzanikolaou2016lightweight, patsakis2015private}, where two parties A and B wants to conclude whether they have the same number privately. The private location equality test can be easily reduced to this problem. The private location equality test, however, does not support comparing with multiple locations, hence can be considered as a very special case of the problem we are studying. 

In order to achieve location security and privacy,  we must perform encryption and decryption methods on the location data, so that users can be secure when servers may be compromised. We adopt well known public-key based encryption methods in our work. Based on the encrypted data, we apply the ideas from the recently proposed homomorphic encryption, which aims to perform complex processing on the encrypted data, yet still yielding results that, once decrypted, are meaningful and correct results on the user side. The ideas of homomorphic computing have been proposed in the literature for several decades, but only in the past several years practical methods have been developed to prove the feasibility of these goals~\cite{Gentry:2009:FHE:1536414.1536440, van2010fully, brakerski2011fully, brakerski2014efficient, brakerski2012leveled, fan2012somewhat, lopez2012fly, brakerski2012fully, bos2013improved, gentry2013homomorphic, brakerski2014lattice}. We therefore build our protocol on top of representative homomorphic computing libraries,  but we note that future developments of better paradigms will lead to lower computing cost and overhead, as well as better security in our system. More specifically, recently there have been two types of homomorphic encryption methods proposed, including the fully homomorphic encryption methods (FHE)~\cite{Gentry:2009:FHE:1536414.1536440, van2010fully, brakerski2011fully} and the somewhat encryption methods~\cite{brakerski2014efficient, brakerski2012leveled, fan2012somewhat, lopez2012fly, brakerski2012fully, bos2013improved, gentry2013homomorphic, brakerski2014lattice}.  While the FHE methods provide support for arbitrary number of operations, the second type only supports a  limited types of operations. Our used libraries fall under the second category, as the FHE methods are still under research due to their excessive demands on computational costs. 
